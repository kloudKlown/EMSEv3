Logs are leveraged by software developers to record and convey important information during the execution of a system. These logs are a valuable source of information for developers to debug large software systems. Prior research has shown that logs are changed during field debugging. However, little is known about how logs are changed during bug fixes. In this paper, we perform a  case study on three large open source platform software namely \emph{ Hadoop, HBase} and \emph{Qpid}. We find that logs are added, deleted and modified statistically significantly more during bug fixes than other code changes. Furthermore, we find identify four different types of modifications that developers make to logs during bug fixes, including: (1)modification to logging level, (2)modification to logging text, (3)modification of logged variable and (4)relocation of log. We find that bug fixes that contain changes to logs have larger code churn, but involve fewer developers, require less time and have less discussion during the bug fix. This suggests that given two bugs of similar complexity, the one which leverages logs and has log changes, has likelihood of being resolved faster.  We build a regression model to explore the relationship between log churn metrics and the resolution time of bugs. We find that log churn metrics can complement traditional metrics, i.e., \# of developers and \# of comments, in explaining the bug resolution . In particular, we find a negative relationship between modifying logs and the resolution time of bugs. This means that bug fixes with log modifications have higher likelihood of being resolved faster.

%Our results suggests that there is a relationship between changing logs and the resolution time of bugs.






%	This shows that logs are helpful in quicker resolution of bugs and demonstrates the importance logs play in bug fixes and motivates developers to log more.
	%	bug fixes with log changes have larger code churn
	
	
	%	Though much research has been done on the analysis of logs, most of
	%	the studies looked at either characterizing logs based on their usage
	%	or how to improve logs so they are more meaningful. But, there has
	%	been no study so far which looks into how effective logs are in the
	%	software development process and especially in debugging. 
	%	
	%	To answer this question in our paper we first try to find co-relation
	%	between log updates and bug fixes. This is an intuitive step, because
	%	when developers fix bugs they update the logs associated with the
	%	bug or add new logs to help them fix bugs. We verify this claim on
	%	3 large scale systems 
	%	We then find the types of changes developers make to logging statements
	%	during bug fixes. Finally, we look at how long it takes for bugs to get fixed,
	%	the number of developers involved in the fix and the comments associated
	%	with the fixes, all with respect to bugs without logging changes.\\
	%	\indent We find that in all projects logging actually helps in the debugging
	%	process and three factors - time, developers and comments - are lesser
	%	when bugs have log churn associated with them.
	
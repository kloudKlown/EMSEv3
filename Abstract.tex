Logs are leveraged by software developers to record and convey important information during the execution of a system. These logs are a valuable source of information for developers to debug large software systems. Prior research has shown that logs are changed during field debugging. However, little is known about how logs are changed during bug fixes. In this paper, we perform a  case study on three large open source platform software namely \emph{Hadoop}, \emph{HBase} and \emph{Qpid}. We find that logs are added, deleted and modified statistically significantly more during bug fixes than other code changes. Furthermore, we identify four different types of modifications that developers make to logs during bug fixes, including: (1) modification to logging level, (2) modification to logging text, (3) modification to logged variable and (4) relocation of log. We find that bug fixes that contain changes to logs have larger code churn, but involve fewer developers, require less time and have less discussion during the bug fix. This suggests that given two bugs of similar complexity, the one which leverages logs and has log changes, has likelihood of being resolved faster. We build a regression model to explore the relationship between log churn metrics and the resolution time of bugs. We find that log churn metrics can complement traditional metrics, i.e., \# of developers and \# of comments, in explaining the bug resolution. In particular, we find a negative relationship between modifying logs and the resolution time of bugs. This means that bug fixes with log modifications have higher likelihood of being resolved faster.


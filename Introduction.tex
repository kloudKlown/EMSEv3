Platform software provides an infrastructure to support a variety of applications. To monitor the execution of these applications
running on top of the platform software, logs are widely leveraged. Such logs are generated through simple \textsl{printf} statements or logging statement in the source code. Logging statements contain a static textual part that provides information about the event, a variable part that contains context of the events, and a logging verbosity level that indicates when the log should be output. An example of a logging statement is shown below where \emph{info} is the logging verbosity level, \emph{Connected to} is the event and the variable \emph{host} contains the information about the logged event. 

\hypobox{ \hspace{25pt} LOG.info( ``Connected to " + host);}


Research has shown that logs are used by developers extensively during the development of software systems~\cite{Characterizinglogs}. Logs are leveraged for anomaly detection~\cite{XUanomalies,ConsoleLogs,Marksyer}, system monitoring~\cite{Bitperformance}, capacity planning~\cite{IanWCRE} and large-scale system testing~\cite{markTesting}. The valuable information in logs has created a new market for log analysis platforms such as Splunk~\cite{Bitperformance}, XpoLog~\cite{Xpolog}, and Logstash~\cite{Logstash}, which assist developers in analyzing the logs.

 
Logs are extensively used to help developers fix bugs in platform software. For example, in the JIRA issue HBASE-3403\footnote{https://issues.apache.org/jira/browse/HBASE-3403}, a bug was reported due to a system failure in the field. Developers leveraged logs to identify the point of failure. After fixing the bug, the logs were updated to prevent similar bugs from re-occurring. A recent study shows that changes to logs have a strong relationship with code quality and that 16-32\%
of log changes are made during field debugging~\cite{EMSEIAN}. However, there exists no large scale study that investigates how logs are changed during bug fixes. Such knowledge may provide guidance for practitioners who often use or change logs. Moreover, the log analysis platform may be improved to better serve the use of logs in practice.

In this paper, we perform an empirical study on the log changes during bug fixes in three open source platform software, i.e., \emph{Hadoop}, \emph{HBase} and \emph{Qpid}. We consider adding, deleting or modifying a logging statements as log changes.\footnote{In the rest of this paper, we use `log' to refer to the logging statements in the source code.} In particular, we sought to answer the following research questions. 

\begin{description}
\item[\textbf{RQ1:}]\textbf{Are logs changed more often during bug fixes?} 

We find that logs are changed more often during bug fixes than non-bug-fixing code changes. In particular, we find that adding and modifying logs occurs more often in bug fixes than non-bug-fixing code changes (statistically significant with non-trivial effect size). We identified four types of modifications to logs, including \emph{modification to logging level}, \emph{modification to logging text}, \emph{modification of logged variable} and \emph{relocation of log}. We find that \emph{modification to logging text}, \emph{modification of logged variable} and \emph{relocation of log} occur more often in bug fixes than non-bug-fixing code changes (statistically significant with medium to high effect sizes). 



\item[\textbf{RQ2:}]\textbf{Is there a relationship between log changes and effort spent on bug fixes?}

We find that bug fixes with log churn have higher total code churn than bug fixes without log churn. After normalizing the code churn, bug fixes with log churn take less time to get resolved, involve fewer developers and have less discussions during the bug fixing process. This suggests that given two metrics of similar complexity the one with log churn is more likely to be resolved faster and involve fewer developers and less discussion. \ian{TODO}


\item[\textbf{RQ3:}]\textbf{Can log churn metrics help in explaining the resolution time of bugs?}

Using log churn metrics (e.g., new logs added, deleted logs) and traditional metrics (i.e., \# comments and \# developers) we trained regression models for the resolution time of bug fixes. We find that modifications to logged variables and modifications to log level are statistically significant in the models and have negative impact on resolution time. This suggests that there is a relationship between these metrics and resolution time of bugs. The relationship shows the impact of log churn on resolution time of bugs which should be further studied. 

\end{description}


The rest of this paper is organized as follows. Section~\ref{sec:Methodology} presents our methodology for extracting data for our study. Section~\ref{sec:Study} presents our case study and the answers to our three research questions. Section~\ref{sec:Related} describes the prior research that is related to our work. Section~\ref{sec:Limit} discusses the threats to validity. Finally, Section~\ref{sec:Conclusion} concludes the paper.



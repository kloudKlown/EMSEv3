\subsection*{RQ2: Is there a relationship between log changes and effort spent on bug fixes?}


\subsubsection*{Motivation}

In RQ1, we find that logs are more likely to be changed in bug fixes. However, we do not know whether changing logs has a relationship with effort spent on bug fixes. If changing logs during bug fixes has relationship with the effort of bug fixes, developers may consider leveraging and improving logs during bug fixes in order to save effort.

\subsubsection*{Approach}


%To answer the research question, 

To study whether there is a relationship between effort spent on bug fixes and changing logs during bug fixes, we collect all JIRA issues with type \emph{bug} from the three studied systems. We obtained the code commits for each of these JIRA issues by searching for the issue id from commit messages. We then separate the JIRA issues into (1) bugs that are fixed with log changes and (2) bugs that are fixed without log changes. We calculate the log churn metrics for each bug. If an issue has multiple commits we consider the code and log changes in all the related commits. We then measure effort of bug fixes using the time taken to resolve a bug, the number of developers involved during the resolution of a bug and the number of discussion posts about the bug on JIRA. Prior research has shown that such metrics are good estimates of developer effort~\cite{amor2006effort}. 

%:\ian{check the paper that you cite to build a model to predict resolution time to see whether there is any thing you can back up for the choice of these three metrics}
\begin{enumerate}
	\item \textbf{Resolution time:} This metric measures how fast the bug is fixed. This metric is defined as the time taken from when the bug is opened until it is resolved. For example, if a bug was opened on 1$ ^{st}$ February 2015 and closed on 5$ ^{th}$ February 2015, the resolution time of the bug is four days. 
	
	\item \textbf {\# of comments:} This metric measures how much discussion is needed to fix a bug. Intuitively, the more discussion in an issue report, the more effort is spent on fixing the bug. We count the total number of comments in the discussion of each issue report.
	
	\item \textbf {\# of developers:} This metric measures the number of developers who participate in the discussion of fixing the bug. Intuitively, more developers who discuss the bug, more effort is spent on fixing the bug. We count the number of unique developers who comment on the issue report. We use the user names in the JIRA discussion to identify the developers. 
\end{enumerate}
Intuitively, a complex bug fix might have more code churn and in turn takes longer time to be resolved, more developers being involved and more discussions on JIRA. Therefore, we use code churn to normalize the resolution time, the number of comments, and the number of developers during bug fixes. We use these normalized effort metrics to find if there is statistically significant difference between bug fixes with log changes and bug fixes without log changes. We use the {\em MannWhitney U} test to find the $|\rho|$ values and \textsl{Cohen's d} test to measure the effect size, similar to RQ1. 

%We use the code churn to measure the complexity of the issue. 


 %and the code churn for fixing each issue We then extracted three metrics from JIRA issues to measure the effort of fixing a bug:

%Prior research has shown that complexity of software can be measured using different metrics, including source lines of code change~\cite{complexity}. 


\begin{table*}
	\caption{Comparing code and developer metrics between the bug fixes with log changes and bug fixes without log changes. The p-value is from MannWhitney U tests and the effect sizes are calculated using Cohen's d. A positive effect size shows that the code changes are more frequent during bug fixes with log changes.}
	\label{tab:bugfixes}
	\centering{}%
	\begin{tabular}{|>{\centering}p{.12\textwidth}|c|>{\centering}p{.1\textwidth}|c|>{\centering}p{.1\textwidth}|c|>{\centering}p{.1\textwidth}|}
		\hline 
		\multirow{2}{*}{Metrics}& \multicolumn{2}{c|}{Hadoop} & \multicolumn{2}{c|}{HBase} & \multicolumn{2}{c|}{Qpid}\tabularnewline
		\cline{2-7} 
		
		& p-values  & Effect size & p-values  & Effect Size & p-values  & Effect size\tabularnewline
		\hline 
		Code churn &  $<$2.2e-16 & 0.178 (small) & $<$2.2e-16 & 0.023 &  $<$2.2e-16 & 0.155\tabularnewline
		\hline 
		Resolution time & 4.7e-14 &  -0.095 & $<$2.2e-16 & -0.188 (small) &  7.7e-08 & -0.276 (small)\tabularnewline
		\hline 
		\# of comments & 2.2e-16 & -0.573 (small) &  $<$2.2e-16 &-0.436 (small) & $<$2.2e-16 & -0.304 (small)\tabularnewline
		\hline 
		\# of developers &  $<$2.2e-16 & -0.539 (small) & $<$2.2e-16 & -0.617 (medium) & $<$2.2e-16 & -0.440 (small)\tabularnewline
		\hline 
	\end{tabular}
\end{table*}
 
 \begin{figure*}[t]
 	\centering
 	\caption{Boxplot of code churn of bug fixes with log changes (shown in blue) against bug fixes without log changes (shown in grey).}
 	\label{fig:figure3}
 	
 	\includegraphics[width=.49\textwidth]{HadoopBoxPlot}
 	\hfill
 	\includegraphics[width=.49\textwidth]{HBaseBoxPlot}\hfill
 	\includegraphics[width=.49\textwidth]{QpidBoxPlot}
 	
 \end{figure*}


\subsubsection*{Results}

\textbf{We find that the log changes are more likely to occur during more complex bugs fixes}. We find that the average code churn for fixing bugs is significantly higher with log changes than without log changes (see Table~\ref{tab:bugfixes} and Figure~\ref{fig:figure3}). This suggests that complex bug fixes are more likely to have log changes (statically significant with non-trivial effect size). 

\textbf{We find that bugs that are fixed with log changes, take shorter time with fewer comments and fewer people.} After normalizing the code churn, we find that the resolution time, the number of comments and the number of developers are all statistically significantly smaller in the bug fixes with log changes than the ones without log changes. This result suggests that given two bugs of the same complexity, the one with log changes is likely to take less time to get resolved and likely to need a fewer number of developers involved with fewer discussions. Changing logs may provide useful information to assist developers in discussing, diagnosing and fixing bugs. For example, when fixing bug HBASE-3074\footnote{https://issues.apache.org/jira/browse/HBASE-30741}, developers suggest providing additional details in the log about where the failure occurs in the JIRA issue report. We find that developers consider this suggestion and add the name of the servers into the logs. Such additional data helps diagnose the cause of the failure and helps fix the bug.

\subsubsection*{Discussion}

To further understand how developers change logs in bug fixes, we finally conduct a qualitative analysis. We collected all the bug fixes with log churn for our studied systems. We selected a 5\% random sample (266 for \textsl{HBase}, 268 for \textsl{Hadoop} and 83 for \textsl{Qpid}) from all the commits. For the sampled commits, we analyze the code changes made in Git and the corresponding JIRA issue reports to find different patterns of log use in bug fixes. We follow an iterative process, similar to prior research~\cite{seaman1999qualitative}, until we cannot find any new types of patterns. We find three reasons of changing logs during bug fixes as shown in Table~\ref{tba:LogUsage}. We find that these three reasons can co-occur within a single commit. 

% The code changes are analyzed to identify common patters of log usage during debugging and JIRA issue reports are analyzed to understand the extent of log usage by developers.
\begin{table}[tbh]
	\protect\caption{Log change reasons during bug fix}
	\label{tba:LogUsage}	
	\begin{centering}
		\begin{tabular}{|>{\centering}p{2.5cm}|>{\centering}p{1.3cm}|>{\centering}p{1.3cm}|>{\centering}p{1.3cm}|}
			\hline 
			Projects & Hadoop & HBase & Qpid\tabularnewline
			\hline 
			\hline 
			Bug diagnosis & 157 & 175 & 49\tabularnewline
			\hline 
			Future bugs prevention & 156 & 170 & 42\tabularnewline
			\hline 
			Code quality assurance for bug fixes & 93 & 78 & 18\tabularnewline
			\hline 
		\end{tabular}
		\par\end{centering}
	
\end{table}
\begin{itemize}
	

\item \subsubsection*{Bug diagnosis} Developers use logs to detect runtime defects. Developers change log to print extra or different information into logs during the execution of the system. The log changes in the category have added and deleted code prior to log changes (i.e., code block is changed). For example, to fix HADOOP-2725 \footnote{https://issues.apache.org/jira/browse/HADOOP-2725} we observe that developers notice a discrepancy when a 100TB file is copied across two clusters. To help in debugging we observe that developers modify the logged variable which outputs the sizes of the files into human readable format instead of bytes. These log changes are committed along with bug fix, as it clarifies the log output and helps in understanding the logging context.

 %These findings are consistent with prior findings where majority of log changes are made during debugging~\cite{EMSEIAN}.

\item \subsubsection*{Future bugs prevention} After fixing a bug, developers may insert log into the code. Such logs monitor the execution of the system to track for similar bugs in the future. Log changes in this category have addition of new blocks (i.e., if, if-else, try-catch and exception) with higher code addition than deletion. For example in HADOOP-2890\footnote{https://issues.apache.org/jira/browse/HADOOP-2890}, we see that developers leverage logs to identify the reason behind blocks getting corrupted. In the commit, we observe that the developers fix this bug by adding new conditionals (i.e., if-else, try-catch, throw) to verify where the block gets corrupted and they add\textsl{try catch} block with new logs to catch these exceptions. The log will notify developers with useful information to diagnose the bug if a similar bug appears.

\item \subsubsection*{Code quality assurance for bug fixes} Sometime, developers need to introduce a large amounts of code to fix a bug with no code deletion. The introduction of new code, may introduce new bugs into the system. To ensure the quality of these bug fixes, developers insert logs into the bug-fixing code. For example in HBASE-3787\footnote{https://issues.apache.org/jira/browse/3787}, where developers encounter a non-idempotent operation (i.e., running the operation more than produces different results) which causes an error in the application. This fix involves over 13 developers and 112 discussions over the two years. The developers add several new files and functions during the bug fix and new logs to assure the code quality of the fix. 

\end{itemize}


\hypobox{Logs are more likely to be changed during complex bugs fixes. After normalizing the complexity of bugs using code churn, we find that bug fixes with log changes are resolved faster with fewer people and fewer discussions.}


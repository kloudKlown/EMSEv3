	In this section, we present our case study results by answering our three research questions. For each research question, we discuss the motivation behind it, the approach to answering the research question and finally the results.

\subsection*{RQ1: Are logs changed more often during bug fixes?}


\subsubsection*{\textbf{Motivation}}

Prior research finds that up to 32\% of the changes to logs are due to field debugging~\cite{EMSEIAN}. During debugging, developers may change logs to gain more run-time information about their system. These log changes may also assist developers in resolving future occurrences of a similar bug. However, to the best of our knowledge, there exists no large scale empirical study to show whether logs are changed more often during bug fixes than other development activities. In addition, we want to investigate how logs are changed during the bug fixes. These findings would provide more insight into what developers consider as important knowledge during bug fixes.

\subsubsection*{\textbf{Approach}}

%Its intuitive that logs are generally modified when bugs are fixed.
%This is because prior research has already proved that a module
%that has been modified continually, has higher chance
%of having bugs than modules which have not been modified \cite{Khosh}.
%So,
We compare the number of changes to logs between bug fixes and non-bug fixes. In previous section, we identified three categories\ian{or types, choose one word across the paper} of logs changes in a commit, i.e., modified, new and deleted logs. Therefore, we compare the number changes to logs within the three types in bug fixes and non-bug fixes. Since commits with higher total code churn may have a higher number of changes to logs, we calculate total code churn for every commit and use it to normalize \emph{\# modified, \# new} and \emph{\# deleted logs}, similar to prior study~\cite{Characterizinglogs}. The three new metrics are:
\begin{equation}
Modified\ log\ ratio = \frac{\#\ of modified\ logs}{\ code\ churn } 
\label{eq1}
\end{equation}
\begin{equation}
New\ log\  ratio = \frac{\#\ of new\ logs}{\ code\ churn } 
\label{eq2}
\end{equation}
\begin{equation}
Deleted\ log\ ratio = \frac{\#\ of deleted\ logs}{\ code\ churn }
\label{eq3} 
\end{equation}

 %Log density is defined as the ratio of total log churn over total code churn across all commits, as used in prior research. We follow the same approach and find log density of each type of log change for bug fixes and non-bug fixes.

To future understand how logs are modified during bug fixes, we perform a manual analysis on the modified logs to identify the different types of log modifications. We first collect all the commits that modify logs. We select a random sample of 357 commits. The size of our random sample achieves a 95\% confidence level and 5\% confidence interval. We follow an iterative process, similar to prior research~\cite{seaman1999qualitative}, to identify the different types of log modifications, until we cannot find any new types of modifications. 

After we identify the types of log modifications, we create an automated tool to label log modifications into the identified types. We calculate the number of log modifications of every type in each commit and normalize the normalize by {\em code churn}, similar to Equation~\ref{eq1} to \ref{eq3}. We call such metrics of log changes, i.e., modified, new, deleted log ratio and the ratios of each type of log modifications, as log churn metrics.
%used  as the controlling measure 

We would like to find out whether logs are changed more during bug fixes than non-bug fixes and to identify the type of log change that developers favor during bug fixes. We first compare the mean value of each log churn metric in bug fixes and non-bug fixes by calculating the ratio between the mean value of each log churn metric in bug fixes and non-bug fixes for every studied project. We further compare the log churn metrics by studying whether there is a statistically significant difference in these metrics, between bug fixes and non-bug fixes. We leverage the \textsl{MannWhitney U test} (Wilcoxon rank-sum test)~\cite{Gehan1965}, since our metrics are highly skewed. The {\em MannWhitney U test} is a non-parametric test, such that it does not have any assumptions about the distribution of the sample population. A p-value of \ensuremath{\le} 0.05 means that the difference of the log churn metrics between the bug fixes and non-bug fixes is statistically significant and we may reject the null hypothesis. By rejecting the null hypothesis, we can accept the alternative hypothesis, which tells us that there is a statistically significant difference for our metrics between bug fixes and non-bug fixes.

%and if we use standard T-test the
%resulting p-value will be wrong. {\em Wilcoxon test} is a non-parametric
%test, meaning the distribution of the population does not factor into
%the results.

We also calculate the {\em effect sizes} in order to quantify the differences in log churn metrics between the bug fixes and non-bug fixes. Unlike the {\em MannWhitney U test}, which only tells us whether the difference between the two distributions is statistically significant, the effect size quantifies the difference between the two distributions. Researchers have shown that reporting only the statistical significance may lead to erroneous results (i.e., if the sample size is very large, the p-value are likely to be small even if the difference is trivial). We use {\em Cohen's d} to quantify the effect size~\cite{cohenUsage1,cohenUsage2}. {\em Cohen's d} is defined as:
\begin{equation} \text{{\em d}} = \frac{\bar{x}_1 - \bar{x}_2}{s},
\label{eq:cohensd}
\end{equation}
where $\bar{x}_1$ and $\bar{x}_2$ are the mean of two populations, $s$ is the pooled standard deviation and $d$ is \emph{Cohen's d}~\cite{shadish2009combining}. We use the following thresholds for {\em Cohen's d}~\cite{Effectsize}:
%As software engineering has different thresholds for {\em Cohen's d}~\cite{Effectsize}, the new scale is shown below.

 \begin{align}\label{cohens-d_interpretation-se}
\begin{cases}
\text{trivial}		& \text{for $d \le 0.17$}		\\
\text{small}		& \text{for $0.17 < d \le 0.6$}	\\
\text{medium}	& \text{for $0.6 < d \le 01.4$}	\\
\text{large}		& \text{for $d > 1.4$}
\end{cases}
\end{align}



\begin{table}[t]
	\protect\caption{Comparing the mean log density between bug fixes and non-bug fixes (the value of 1.9 for Hadoop means logs are modified 1.9 times more during bug fixes compared to non-bug fixes).}
	\begin{tabular}{|l|lll|}
		\hline 
		\multirow{1}{*}{Projects} & \multicolumn{1}{c}{Modified log density ratio} & \multicolumn{1}{c}{New log density ratio} & \multicolumn{1}{c|}{Deleted log density}\tabularnewline
		\hline 
		\multirow{1}{*}{Hadoop} & 1.9 & 1.4 & 1.6\tabularnewline
		%		\hline 
		\multirow{1}{*}{HBase} & 3.5 & 1.2 & 2.4\tabularnewline
		%		\hline 
		\multirow{1}{*}{Qpid} & 4.1 & 1.2 & 0.5\tabularnewline
		\hline 
	\end{tabular}
	\label{tba:logdensityNewLogs}
\end{table}


\begin{table}[tbh]
	\caption{Comparing log churn metrics between the bug fixes and non-bug fixes. The p-value is from MannWhitney U tests and the effect sizes are calculated using Cohen's d. A positive effect size shows that the log changes are more frequent during bug fixes and P-values are smaller than 0.05.}
	\label{tab:logchange} 
	\centering{}%
	\begin{tabular}{|>{\centering}p{.13\textwidth}|>{\centering}p{.1\textwidth}|>{\centering}p{.1\textwidth}|>{\centering}p{.12\textwidth}|>{\centering}p{.1\textwidth}|>{\centering}p{.1\textwidth}|>{\centering}p{.1\textwidth}|}
		\hline 
		\multirow{2}{*}{Metrics}& \multicolumn{2}{c|}{Hadoop} & \multicolumn{2}{c|}{HBase} & \multicolumn{2}{c|}{Qpid}\tabularnewline
		\cline{2-7} 
		
		& P-Values & Effect Size & P-Values & Effect Size & P-Values & Effect Size\tabularnewline
		\hline 
		Modified logs  ratio & \textbf{2.0e-12} & 0.246 (small) & \textbf{1.9e-15} & 0.273 (small) & \textbf{1.6e-11} &  0.432 (small)\tabularnewline
		\hline 
		New logs ratio& \textbf{ 4.7e-16} &  0.265 (small) & \textbf{$<$2.2e-16} & 0.215 (small) & \textbf{ 2.1e-11} & 0.474 (small)\tabularnewline
		\hline 
		Deleted logs ratio& \textbf{8.1e-07} & 0.336 (small) & \textbf{4.9e-07} & 0.150 & \textbf{0.041} & -0.193 (small)\tabularnewline
		\hline 
	\end{tabular}
	
\end{table}

\subsubsection*{\textbf{Results}}
\textbf{Developers are more likely to add new logs when fixing bugs.} From Table~\ref{tba:logdensityNewLogs}, we find that the mean of new log density is 1.2-1.4 times in bug fixes to non-bug fixes. This shows that new logs are 20\%-40\% more likely to occur in bug fixes that non-bug fixes. Table~\ref{tab:logchange} shows that $new\ logs\ ratio$ in bug fixes is higher than non-bug fixes in all studied systems (statistically significant with non-trivial effect sizes). This suggest that developers may need additional information during bug fixes and they add new logs to help fix bugs.
%
%This suggests that developers add new logs during bug fixes than non bug fixes. We find that effect size of new log ratio is higher in \emph{Qpid} than \emph{Hadoop} and \emph{HBase}. This may be because the new log ratio of bug fixes, is three times that of non-bug fixes in \emph{Qpid}. In \emph{Hadoop} and \emph{HBase} we find new log ratio of bug fixes to be only twice that of non-bug fixes. This may be because \emph{Qpid} being a relatively newer system, some important source code may not be well logged. Therefore, developers may have to add additional logs to assist in bug fixes and improve code quality~\cite{EMSEIAN}.

\textbf{Developers are less likely to remove logs during bug fixes.} We find that although the difference of $deleted\ logging\ statements\ ratio$ between bug fixes and non-bug fixes is statistically significant in all projects, the effect sizes is trivial for \emph{HBase} and negative for \emph{Qpid} (see Table~\ref{tab:logchange}). Moreover, we find that the log density varies between 0.5-2.4. This suggests developers are more likely to remove logs during bug fixes in \emph{Qpid}. Such results confirm the findings from prior research that deleted logs do not have a strong relationship with code quality~\cite{IanWCRE}. 


\begin{table}[t]
	\caption{Distribution of four types of log modifications.}	
	\label{tab:dist}
	\centering
	\begin{tabular}{|>{\centering}p{2.2cm}|>{\centering}p{1.3cm}|>{\centering}p{1.3cm}|>{\centering}p{1.3cm}|}
		\hline 
		Projects & \multicolumn{1}{c|}{Hadoop (\%)} & \multicolumn{1}{c|}{HBase (\%)} & \multicolumn{1}{c|}{Qpid (\%)}\tabularnewline
		\hline 
		Logging statement relocation  & 73.1 & 70.7 & 47.4\tabularnewline
		\hline 
		Text Modification & 10.5 & 13.4 & 16.8\tabularnewline
		\hline 
		Variable Modification & 9.9  & 10.1 & 18.9\tabularnewline
		\hline 
		Logging Level Change & 6.5 & 5.8  & 16.8\tabularnewline
		\hline 
	\end{tabular}
\end{table}


\textbf{Logging statements are more likely to be modified in bug fixes and non-bug fixes}. From Table~\ref{tba:logdensityNewLogs} we find that logs are 1.9-4.1 times more likely to be modified during bug fixes than non bug fixes. Table~\ref{tab:logchange} shows that $modified\ log-\ $ 
$ratio$ in bug fixes is higher than non-bug fixes in all studied systems (statistically significantly with non-trivial effect sizes). Such results show that developers often change the information provided by logs to assist in bug fixes. Developers may need different information to the information that is provided by the logs to fix the bugs. Prior research find that 66 \% of the logs are modified when 1) the condition the log code depends on is changed, 2) the logged variable is changed or 3) the function name which is also referred in static text of log is modified~\cite{Characterizinglogs}. Therefore, we further explore the different types of modifications to logs.

We manually look at the different modifications to logs and categorize the modifications into four types. They are described below and Table~\ref{tab:dist} shows the distribution of each type in our studied systems. When there is co-occurrence of different types of log modifications in a single log, we treat that as new log because it is difficult to categorize all the different types of co-occurrences. 

\begin{enumerate}
	\item \textbf{Logging statement relocation:} The log is not changed but moved to a different place in the file.
	\item \textbf{Text modification:} The text that is printed in the logs is modified.
	\item \textbf{Variable change:} One or more variables in the logs are changed (added, deleted or modified).
	\item \textbf{Logging level change:} The verbosity level of logs are changed.
\end{enumerate}



\textbf{Developers modify logged variables during bug fixes}. From Table~\ref{tba:logdensityNewLogs} we find that variables in logs are 1.7-4.4 times more likely to be modified during bug fixes. We find that changes to the logged variables is higher in bug fixes and non-bug fixes as seen in Table~\ref{tab:logmod} (statistically significant with medium or small effect sizes). This suggests that developers change the logged variables in order to provide useful information in the log outputs. Prior research also shows that 16-32\% of log changes are made during field debugging~\cite{EMSEIAN}. Therefore, to better understand how developers change logged variables during bug fixes, we categorize the changes into three types: a) variable addition, b) variable deletion and c) variable modification.
\begin{table}[t]
	\protect\caption{Comparing the mean log density between bug fixes and non-bug fixes. }
	
	\centering
	\begin{tabular}{|>{\centering}p{.15\textwidth}|>{\centering}p{.15\textwidth}|>{\centering}p{.15\textwidth}|>{\centering}p{.15\textwidth}|>{\centering}p{.15\textwidth}|}
		\hline 
		\multirow{1}{*}{Projects} & Log relocation density ratio & Text modification density ratio & Variable modification density ratio & Logging level density ratio\tabularnewline
		\hline 
		\multirow{1}{*}{Hadoop} & 2.0 & 2.5 & 2.0 & 4.0\tabularnewline
%		\hline 
		\multirow{1}{*}{HBase} & 3.3 & 6.5 & 1.7 & 7.5\tabularnewline
%		\hline 
		\multirow{1}{*}{Qpid} & 4.9 & 11.1 & 4.4 & 15.8\tabularnewline
		\hline 
	\end{tabular}
\end{table}


\begin{table}[t]
	\protect\caption{Comparing logging modification metrics between the bug fixes and non-bug fixes . The p-value is from MannWhitney U tests and the effect sizes are calculated using Cohen's d. A positive effect size shows that the log changes are more frequent during bug fixes and p-values are smaller than 0.05}
	\label{tab:logmod}
	\centering{}%
	\begin{tabular}{|>{\centering}p{.13\textwidth}|c|>{\centering}p{.1\textwidth}|c|>{\centering}p{.1\textwidth}|c|>{\centering}p{.1\textwidth}|}
		\hline 
		\multirow{2}{*}{Metrics}& \multicolumn{2}{c|}{Hadoop} & \multicolumn{2}{c|}{HBase} & \multicolumn{2}{c|}{Qpid}\tabularnewline
		\cline{2-7} 
		& P-values  & Effect sizes & P-values  & Effect sizes & P-values  & Effect sizes\tabularnewline
		\hline 
		Relocation of log & \textbf{1.1e-10} & 0.330 (small) & \textbf{3.0e-11} & 0.170 (small) & \textbf{1.8e-08} & 0.700 (medium)\tabularnewline
		\hline 
		Modification to logging text & 0.344 & - & \textbf{0.0075} & 0.525 (small) & \textbf{4.5e-06} & 0.976 (medium)\tabularnewline
		\hline 
		Modification of logged variable  & \textbf{1.3e-04} &0.351 (small) & \textbf{0.0010} & 0.420 (small) & \textbf{1.2e-04} &  1.17 (medium)\tabularnewline
		\hline 
		Modification to logging level &  0.108 & - & 0.503 & - & 0.399 & - \tabularnewline
		\hline 
	\end{tabular}
\end{table}



\begin{table}[t]
	\caption{Comparing the different types of changes to variables between the bug fixes and non-bug fixes. The p-value is from MannWhitney U tests and the effect sizes are calculated using Cohen's d. A positive effect size shows that the log changes are more frequent during bug fixes and p-values are smaller than 0.05}
	\label{tab:varmod}
	\centering
	\begin{tabular}{|>{\centering}p{.13\textwidth}|c|>{\centering}p{.1\textwidth}|c|>{\centering}p{.1\textwidth}|>{\centering}p{.1\textwidth}|p{0.1\textwidth} |}
		\hline 
		\multirow{2}{*}{Metrics}& \multicolumn{2}{c|}{Hadoop} & \multicolumn{2}{c|}{HBase} & \multicolumn{2}{c|}{Qpid}\tabularnewline
		\cline{2-7} 
		& P-values  & Effect sizes & P-values  & Effect sizes & P-values  & Effect sizes\tabularnewline
		\hline  Variable addition & 0.099  & -  & \textbf{0.00049}& 0.659 (medium) & \textbf{0.005}& 1.40 (large) \\ 
		\hline  Variable deletion & 0.098 &   - & 0.355 & -  & 0.193 & -  \\ 
		\hline Variable modification & \textbf{4.11e-05} & 1.045 (medium)  & 0.568 & - & \textbf{0.0016}& 0.949 (medium)   \\ 
		\hline 
	\end{tabular} 
\end{table}

\begin{table}[t]
	\protect\caption{Comparing the log density between bug fixes and non-bug fixes. }
	
	\centering
	\begin{tabular}{|>{\centering}p{.15\textwidth}|>{\centering}p{.15\textwidth}|>{\centering}p{.15\textwidth}|>{\centering}p{.15\textwidth}|}
		\hline 
		\multirow{1}{*}{Projects} & Variable addition density ratio & Variable deletion density ratio & Variable modification density ratio \tabularnewline
		\hline 
		\multirow{1}{*}{Hadoop} & 2.7 & 2.6 & 2.9 \tabularnewline
		%		\hline 
		\multirow{1}{*}{HBase} & 1.3 & 1.6 & 0.8 \tabularnewline
		%		\hline 
		\multirow{1}{*}{Qpid} & 25.2 & 4.6 & 4.7 \tabularnewline
		\hline 
	\end{tabular}
\end{table}

Table~\ref{tab:varmod} shows that developers modify the logged variables more in bug fixes than non-bug fixes(statistically significant with medium effect sizes). This modification may be because developers need different information in the output logs, than provided by existing logs. For example, when manually exploring the patch notes for bug QPID-2370\footnote{https://issues.apache.org/jira/browse/QPID-2370}, we find that developers modify the existing log to capture the newly defined variable. The other reason may be that developers change the name of the existing variable to a more meaningful name. For example, in bug MAPREDUCE-2264\footnote{https://issues.apache.org/jira/browse/MAPREDUCE-2264}, we find that developers rename variables and modify the logs accordingly. %Prior research also find that developers often have after-thoughts on logs~\cite{Characterizinglogs}.

From Table~\ref{tab:varmod}, we observe that the developers add variables more in bug fixes than non-bug fixes(statistically significant with medium effect size in \emph{HBase} and large effect size in \emph{Qpid}). We find from Table~\ref{tba:logdensityNewLogs} that developers are 1.3-25.2 times more likely to add new variables during bug fixes.  This addition of variables suggests, existing variables may not have all the needed information in the log output and developers have to add new variables during bug fixes. We also find that developers do not delete variables in logs. Deleting variables in logs may change the format of the output logs. There may be log processing tools that rely on these output logs. Deleting variables may impact the correctness of these log processing applications~\cite{IanWCRE}. Therefore, developer may be award of this and try to avoid deleting variables from logs.
 
 
\textbf{Developers modify logged text more during bug fixes.} We find that modification of logged text is higher in bug fixes and non-bug fixes with non-trivial effect sizes (see Table~\ref{tab:logmod}). In some cases, the text description in logs is not clear and developers need to improve the text to provide more clarity. For example, in \emph{HBase} HBASE-6665\footnote{https://issues.apache.org/jira/browse/HBASE-6665} developers modify the log to provide more information about which region (i.e., Root or Meta) is being used in the operation. Prior research shows that up-to 39\% of modifications to logged text is due to inconsistency between the execution event and the message conveyed by log output~\cite{Characterizinglogs}. Our results suggest that developers may have faced such challenges during bug fixes and may have modified the text in logs for clarification. 
%'

%that developers value more in providing contextual data to an existing
%log rather than writing new logs. 

\textbf{Logging statement relocation occurs more in bug fixes.} Table~\ref{tab:dist}, shows that there are a large number of logging changes that only relocate logs. Table~\ref{tab:logmod} shows that such relocation of logs is statistically significantly more in bug fixes and non-bug fixes (2.0-4.9 times more likely in bug fixes as shown in Table~\ref{tba:logdensityNewLogs}). We manually examine such commits and find that developers often forget to leverage exception handling or using proper condition statements in the code. After fixing the bugs, developers often move existing logs into the \emph{try/catch} blocks or after condition statements. For example, in the YARN-289 \footnote{https://issues.apache.org/jira/browse/YARN-289} of Hadoop, logs are placed into the proper \emph{if-else} block.

\textbf{Logging levels are not modified often during bug fixes.} We find that logging level changes are statistically indistinguishable betweenbug fixes and non-bug fixes in all studied systems. The reason may be that developers are able to enable all the logs during bug fixes, despite of what level a log has. In addition, prior research shows that developers do not have a good knowledge about how to choose a correct logging level~\cite{Characterizinglogs}. 


\hypobox{Developers change logs statistically significantly more in bug fixes than non-bug fixes in a given file. In particular, developers modify logs to add or change the variables in logs during bug fixes. This suggests that developers often realize the needed information to be logged as after-thoughts and change the variables in log to assist in fixing bugs.}

